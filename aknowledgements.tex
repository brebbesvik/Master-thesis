This thesis wouldn't been possible without the effort and contribution of some of the people I have had the pleasure to work with the last 18 months.

My supervisors Yngve Lamo and Svein Ivar Lillehaug which have contributed with valuable help, feedback and ideas in periodic meetings, arranging workshop and introducing me for people which have helped me under way.
 
Job Nyangena has been my closest collaborator, and have contributed a lot with his medical knowledge, ideas and feedback on my work. Our many discussions have had a great impact on the direction of the project.

Fazzle Rabbi has been of great help with his genuine interest for the project. He has always been available when I've needed help with challenges in model driven engineering.

A big thanks to Rosaline Barendregt which contributed with her knowledge and experience in interaction design and gamification. 

Idar Syslak which helped me comparing similar products, as well as coming up with ideas and helping me design alternative prototypes for guideline training.

August Hoel and Fredrik Hoel graduated as medical doctors and started a career in research during my masters degree project. They spent many hours helping me discussing the role of clinical practice guidelines in medical training for students, and how the guidelines are used in the daily practices of health workers. They also spent hours evaluating and discussing the final version of the serious game for guideline training.

Thomas Berge for using his time to evaluate the final version of the serious game for guideline training. He introduces me to some of his colleagues at polyclinic for pulmonary diseases at Haukeland University, which also contributed to the evaluation.

A great thanks to Nikolai Grieg and Gard Engen for moral support, their contribution to a great daily work environment and subject-related input and discussions.

To Malin, my parents, my brother and his family for their patience, support and understanding during this period. I wish I had more time to spend with you all. 





 


 