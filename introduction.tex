
\section{Paper publication}
In December 2018, we submitted a paper "A Model Driven Approach to the Development of Gamified Interactive Clinical Practice Guidelines", which was a summary of our work so far in this project and related projects. The paper was accepted for publication by ENASE 2019 – 14th International Conference on Evaluation of Novel Approaches to Software Engineering, where we held a presentation on their conference in Heraklion, Greece. 

The paper can be found in appendix \ref{appendix:Paper} in this thesis.

%\section{Old Research questions}
%\begin{itemize}
	%\item Can we make a data structure representing the paediatric possible asthma guideline \parencite{RepublicofKeny2016}  in a very generic way?
	%\item Based on the data structure, can we generate suitable scenarios, with multiple choice %question with answer elements for training and evaluating health personnel?
	%\item How can we structure the learning material to best train medical students, doctors, clinical officers, nurses and other health workers in the paediatric possible asthma guideline \parencite{RepublicofKeny2016}? 
			%\item Based on clinical guidelines, can we make a reusable data structure representing respiratory diseases for use in serious games?


%-------
%	\item Based on clinical guidelines, can we make a data structure which is easy to implement in the system, as well as adaptable? 
	
%	\item How to use such a model for generating and testing case based multiple choice questions and answer elements?
	%\item Can we use the data model to structure the learning content such that it is adapted to the current knowledge of the individual learner?
	
	
%	\item How can we model the work-flow of a clinical encounter, a patient at a given point in the clinical encounter, and a student at the current point in his learning process. How to represent these?	
%\end{itemize}

%\textcolor{purple}{TODO Yngve: Gamification, hva evaluerer du?}


\section{Research questions}

\begin{itemize}
	\item \textbf{RQ1:} Based on clinical guidelines, how can we define and represent a generic data structure that can be used to implement applications such as online guidelines or training games for such guidelines, and where applications can adapt to the level of their users?
	\item \textbf{RQ2:} Can the generic data structure in RQ1 be used to generate a specific data model for another domain such as paediatric asthma?
	\item \textbf{RQ3:} How can we use the data model in RQ2 to implement a game for guideline training that can adapt to the level and progression  of users?
	\item \textbf{RQ4:} Is the guideline meta model at an abstraction level such that it can be used for other guidelines? 
\end{itemize}
\section{Structure of the thesis}
\begin{enumerate}
	\item \textbf{Background:} in this chapter we give an explanation of what the clinical practice guidelines are and the purpose they serve. We propose a serious game which will address some of the reasons why clinical practice guidelines haven't been put more into use.
	\item \textbf{Introduction:} we presents some research questions which are related to making abstract and specific data structures of the guideline content, as well as representing the guideline content at the knowledge level of the student, adapt it during the student's progression and make it flexible such that the student can pick his own path through the learning material.
	\item \textbf{Method:} in this chapter we describe our adoption to design science and go through the seven guidelines. The concern that we must deliver value to the medical community, as well as a contribution to the science of health informatics. We describe our work as iterations with specific goals and evaluations.
	\item text{Architecture:} we describe the overall architecture of the game. We have a presentation layer, where we discuss all the technologies in that layer. We have a separate game engine, which uses four managers and three conceptual managers. The managers describes the responsibilities of the game engine. This is a brief introduction, and more detail about the game engine will be discussed in the following chapters.
	\item \textbf{Data models:} We give a short explanation of model driven engineering and DPF.We proposed four data models, an entity model, a workflow model, a game model and student learning model. We will cover the models in detail in this chapter.
	\item \textbf{Game elements:} We will discuss the conceptual question flow manager, conversation manager and user manager. We will use Dynamic Content Management to make the learning material adaptable to the knowledge level and progression of the student, as well as flexigle such that the student can choose his own way through the learning content. We have to expand the entity model with presentation vertices, such that we can produce textual questions. We discuss the multiple-try feedback and other quiz game approaches, as well as a rewarding system. We show a conceptual model of the game engine.
	\item \textbf{Application walktrough:} we play through level 2, presents screenshots and discuss features, ideas, user interface and design choices.
	\item \textbf{Evaluation:} we evaluate the data models by modelling the paediatric pneumonia guideline \parencite{RepublicofKeny2016}. We evaluate the game itself with two nurses and two medical doctors. We discuss our findings in relation to the research questions.
	\item \textbf{Discussion and conclusion:} we briefly discuss topics which may be done in relation to this project as future work. We present some related work and then concludes this thesis.
\end{enumerate}
\section{Summary}
Here we have defined a set of research questions, which is related to the development of serious games for clinical guidelines. Making games which are adaptable to the knowledge level, progression of the user, and making guideline models which are at an abstraction level where they can be used to represent other CPGs, are the main focus points.

We have also given a short presentation of each chapter in the thesis.

