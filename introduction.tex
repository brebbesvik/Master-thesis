
\section{Motivation}
\section{Old Research questions}
\begin{itemize}
	%\item Can we make a data structure representing the paediatric possible asthma guideline \parencite{RepublicofKeny2016}  in a very generic way?
	%\item Based on the data structure, can we generate suitable scenarios, with multiple choice %question with answer elements for training and evaluating health personnel?
	%\item How can we structure the learning material to best train medical students, doctors, clinical officers, nurses and other health workers in the paediatric possible asthma guideline \parencite{RepublicofKeny2016}? 
			%\item Based on clinical guidelines, can we make a reusable data structure representing respiratory diseases for use in serious games?
	\item Based on clinical guidelines, can we make a data structure which is easy to implement in the system, as well as adaptable? 
	
	\item How to use such a model for generating and testing case based multiple choice questions and answer elements?
	%\item Can we use the data model to structure the learning content such that it is adapted to the current knowledge of the individual learner?
	
	
	\item How can we model the work-flow of a clinical encounter, a patient at a given point in the clinical encounter, and a student at the current point in his learning process. How to represent these?	
\end{itemize}
\textcolor{purple}{TODO Yngve: Gamification, hva evaluerer du?}
\section{New Research questions}
\begin{itemize}
	\item \textbf{RQ1:} Based on clinical guidelines, how can we define and represent a generic data structure that can be used to implement applications such as online guidelines or training games for such guidelines, and where applications can adapt to the level of their users?
	\item \textbf{RQ2:} Can the generic data structure in RQ1 be used to generate a specific data model for another domain such as paediatric asthma?
	\item \textbf{RQ3:} How can we use the data model in RQ2 to implement a game for guideline training that can adapt to the level and progression  of users?
	\item \textbf{RQ4:} Is the guideline meta model at an abstraction level such that it can be used for other guidelines? 
	\item \textbf{RQ5:}
\end{itemize}
\section{Structure of the thesis}