\documentclass[a4paper,12pt]{book}
\usepackage[utf8]{inputenc}
%\usepackage[margin=24mm]{geometry}
%\usepackage[top=1in, bottom=1in, left=1.25in, right=1.25in]{geometry}
%\usepackage[top=1in, bottom=1in, left=1.5in, right=1.25in]{geometry}


\usepackage{float}
\usepackage{graphicx}
\graphicspath{ {images/} }

\usepackage[
backend=bibtex,
style=authoryear,
sorting=nty,
citestyle=authoryear,
autocite=inline
]{biblatex}

\addbibresource{references/references.bib}

\usepackage{helvet}
\usepackage{subfig}

\usepackage[Lenny]{fncychap}
\usepackage{xcolor}
\usepackage{array}
\usepackage{listings}

\usepackage{amssymb}


\usepackage{pdfpages}
\usepackage[toc,page]{appendix}


\begin{document}

\frontmatter
%----------------------------------------------------------------------------------------
%	TITLE PAGE
%----------------------------------------------------------------------------------------

\newcommand*{\titlePage}{\begingroup % Create the command for including the title page in the document
	\fontfamily{phv}\selectfont
	\centering % Center all text
	
	\vspace{200pt}
	{\Huge Gamification To Promote Guideline Training In Health Care} \\ % Title
	\vspace{5pt}
	
	%{\Large \textsl{Design science}} % Subtitle or further description
	\vspace{50pt}
	
	{\Large{Ben-Richard Sletten Ebbesvik}}\\ % Author name
	
	\vfill % Whitespace between the author name and the publisher logo
	
	{\large Master thesis in Software Engineering at \\
		\vspace{10pt}
		Department of Computing, Mathematics and Physics, \\
		Western Norway University of Applied Sciences \\
		\vspace{10pt}
		Department of  Informatics, \\
		University of Bergen \\}
	\vspace{10pt}
	{\large Supervisors: Yngve Lamo and Svein Ivar Lillehaug \\}
	\vspace{10pt}
	{\large June 2019} % Month and year published
	
	\begin{figure}[h!]
		\captionsetup[subfigure]{labelformat=empty}
		\subfloat[][]{\includegraphics[width=250pt]{images/hvl_logo_engelsk.pdf}}
		\hfill
		\subfloat[][]{\includegraphics[width=70pt]{images/uib-logo.pdf}}
	\end{figure}
	
	
	\endgroup}
\titlePage

\tableofcontents
\mainmatter


\section{Paper publication}
In December 2018, we submitted a paper "A Model Driven Approach to the Development of Gamified Interactive Clinical Practice Guidelines", which was a summary of our work so far in this project and related projects. The paper was accepted for publication by ENASE 2019 – 14th International Conference on Evaluation of Novel Approaches to Software Engineering, where we held a presentation on their conference in Heraklion, Greece. 

The paper can be found in appendix \ref{appendix:Paper} in this thesis.

%\section{Old Research questions}
%\begin{itemize}
	%\item Can we make a data structure representing the paediatric possible asthma guideline \parencite{RepublicofKeny2016}  in a very generic way?
	%\item Based on the data structure, can we generate suitable scenarios, with multiple choice %question with answer elements for training and evaluating health personnel?
	%\item How can we structure the learning material to best train medical students, doctors, clinical officers, nurses and other health workers in the paediatric possible asthma guideline \parencite{RepublicofKeny2016}? 
			%\item Based on clinical guidelines, can we make a reusable data structure representing respiratory diseases for use in serious games?


%-------
%	\item Based on clinical guidelines, can we make a data structure which is easy to implement in the system, as well as adaptable? 
	
%	\item How to use such a model for generating and testing case based multiple choice questions and answer elements?
	%\item Can we use the data model to structure the learning content such that it is adapted to the current knowledge of the individual learner?
	
	
%	\item How can we model the work-flow of a clinical encounter, a patient at a given point in the clinical encounter, and a student at the current point in his learning process. How to represent these?	
%\end{itemize}

%\textcolor{purple}{TODO Yngve: Gamification, hva evaluerer du?}


\section{Research questions}

\begin{itemize}
	\item \textbf{RQ1:} Based on clinical guidelines, how can we define and represent a generic data structure that can be used to implement applications such as online guidelines or training games for such guidelines, and where applications can adapt to the level of their users?
	\item \textbf{RQ2:} Can the generic data structure in RQ1 be used to generate a specific data model for another domain such as paediatric asthma?
	\item \textbf{RQ3:} How can we use the data model in RQ2 to implement a game for guideline training that can adapt to the level and progression  of users?
	\item \textbf{RQ4:} Is the guideline meta model at an abstraction level such that it can be used for other guidelines? 
	\item \textbf{RQ5:}
\end{itemize}
\section{Structure of the thesis}
This thesis is divided into nine chapters. This section gives a brief overview of the content in each chapter.
\begin{enumerate}
	\item \textbf{Background:} in this chapter we give an explanation of what the clinical practice guidelines are and the purpose they serve. We propose a serious game which will address some of the reasons why clinical practice guidelines haven't been put more into use.
	
	\item \textbf{Introduction:} we presents some research questions which are related to making abstract and specific data structures of the guideline content, as well as representing the guideline content at the knowledge level of the student, adapt it during the student's progression and make it flexible such that the student can pick his own path through the learning material.
	
	\item \textbf{Method:} in this chapter we describe our adoption to design science and go through the seven guidelines. The concern that we must deliver value to the medical community, as well as a contribution to the science of health informatics. We describe our work as iterations with specific goals and evaluations.
	
	\item \textbf{Architecture:} we describe the overall architecture of the game. We have a presentation layer, where we discuss all the technologies in that layer. We have a separate game engine, which uses four managers and three conceptual managers. The managers describes the responsibilities of the game engine. This is a brief introduction, and more detail about the game engine will be discussed in the following chapters.

	\item \textbf{Data models:} We give a short explanation of model driven engineering and DPF.We proposed four data models, an entity model, a workflow model, a game model and student learning model. We will cover the models in detail in this chapter.
	\item \textbf{Game elements:} We will discuss the conceptual question flow manager, conversation manager and user manager. We will use Dynamic Content Management to make the learning material adaptable to the knowledge level and progression of the student, as well as flexible such that the student can choose his own way through the learning content. We have to expand the entity model with presentation vertices, such that we can produce textual questions. We discuss the multiple-try feedback and other quiz game approaches, as well as a rewarding system. We show a conceptual model of the game engine.

	\item \textbf{Application walktrough:} we play through level 2, presents screenshots and discuss features, ideas, user interface and design choices.

	\item \textbf{Evaluation:} we evaluate the data models by modelling the paediatric pneumonia guideline \parencite{RepublicofKeny2016}. We evaluate the game itself with two nurses and two medical doctors. We discuss our findings in relation to the research questions.

	\item \textbf{Discussion and conclusion:} we briefly discuss topics which may be done in relation to this project as future work. We present some related work and then concludes this thesis.
\end{enumerate}
\section{Summary}
Here we have defined a set of research questions, which is related to the development of serious games for clinical guidelines. Making games which are adaptable to the knowledge level, progression of the user, and making guideline models which are at an abstraction level where they can be used to represent other CPGs, are the main focus points.

We have also given a short presentation of each chapter in the thesis.


\chapter{Background}
\section{Clinical Practice Guidelines}
For a clinician, \textcite{Fervers2010} claims that increased medical knowledge is associated with an exponential growth of scientific data and published material. It is impossible to keep up, as well as integrating all the new information into daily practice to give patients the best possible care.  \textcite{Masic2008} gives an example where a general practitioner should read 19 articles per day to keep up with the new medical information, while only having time to read one hour per week. The clinician will spend more time reading articles than treating patients. This problem is known as academic isolation.

Evidence Based Medicine (EBM) suggests that instead of routinely reading dozens of articles, the clinicians should target their reading to specific patient problems. Developing clinical questions and then searching for the answer (problem based approach), may be a more productive way to keep up with the new medical knowledge \parencite{Masic2008}. The EBM definition further puts an emphasize on integrating the best evidence in decision making with the clinicians expertise and the patients values and expectations \parencite{Masic2008}. 

The concept of EBM is about transferring knowledge from clinical research into clinical practice, and Clinical Practice Guidelines (CPG) can play an instrumental role in this process \parencite{Fervers2010}.

The Institute of Medicine (IOM) has given the following definition of clinical practice guidelines: "CPGs are statements that include recommendations intended to optimize patient care. These statements are informed by a systematic review of evidence and an assessment of the benefits and costs of alternative care options" \parencite{Guidelines2011}

The definition given by IOM covers the goals in EBM, and also takes the cost into account. In fact, \textcite{Clayton1995} have shown that in some situations good use of appropriate guidelines and protocols can reduce as much as 25\% of the cost of healthcare.

% I can use Woolfs article and write even more about benefits. Should probably do that.

Even though the CPGs have proven to improve the quality of health care while reducing practice variability and the cost of patient care \parencite{DeClercq2008}, it is well recognized that CPGs have had a limited effect on changing the clinicians practice methods. \textcite{Cabana1999} lists the following reasons:
\begin{itemize}
	\item Lack of awareness.
	\item Lack of familiarity.
	\item Lack of agreement with the content.
	\item Lack of self-efficacy.
	\item Lack of outcome expectancy.
	\item Inertia of previous practice. 
	\item External Barriers; the guidelines are not easy to use, not convenient, cumbersome and confusing.
\end{itemize}One example of external barrier is the Guidelines for the Diagnosis and Management of Asthma \parencite{NationalHeartLungandBloodInstitute2007}, which consists of 440 pages. Such a large journal is not convenient to use at the point of care. According to \textcite{Shortliffe1998}, CPGs in monographs and journal articles tend to sit on book shelves at the time their knowledge could prove the most valuable to the clinicians. 

According to \textcite{Woolf1999}, clinicians sometimes have good reasons to disagree with some of the content of a guideline. \textcite{Woolf1999} points out three reasons:
\begin{enumerate}
	\item The scientific evidence of the recommendation can be lacking, misleading or misinterpreted.
	\item The recommendations may be influenced by the authors. What the authors believe, may be inferior to other options, ineffective or harmful.
	\item As the guideline may be written to control cost, serve societal needs or protect special interest, the recommendations may be suboptimal for the patient.
\end{enumerate}
There exists grading systems which grade the quality of evidence and strength of recommendations. GRADE is such a grading system \parencite{Guyatt2008}.


\textcolor{red}{\begin{itemize}
	\item Medical knowledge increases. Hard to keep track
	\item Guidelines is a summary of the available evidence of the medical conditions and provide management and recommendations
	\item A well-developed guideline reduces
	variations in care, improves diagnostic accuracy,
	promotes effective therapy and discourages ineffective
	therapies all which contribute to improved
	quality of care (citation)
	\item The CPGs are not used enough
	\item Dissemination and implementation
	\item Large volume of excisting guidelines. Difficult to use at the point of care
	\item Dissemination
	\item Different practice even in the same country
\end{itemize}}
\section{Serious games}
\section{Asthma}
\subsection{Challenges}
\section{Related work}
\section{Summary}


\chapter{Method}
\section{Design study}

\subsection{Workshop}
The 22nd of February 2019 we had a workshop. The purpose of the workshop was to
\begin{itemize}
	\item Identify components in the treatment plan of asthma patients.
	\item Identify difficulty levels, and how the questions will be more detailed for every difficulty level.
	\item Make a map of the learning content. Where the content is categorized in components and difficulty levels.  Identify paths the student can take through the learning content.f 
\end{itemize}

\textcolor{purple}{TODO: Move names over in aknowledgements}
The antecedences for the meeting was 
\begin{itemize}
	\item Professor in computer science. Background in model driven engineering and health informatics.
	\item Assistant professor in computer science. Background as a researcher in health informatics.
	\item Postdoctoral fellow. Background in model driven engineering.
	\item Medical doctor and PhD student in health informatics.
	\item PhD research fellow in interaction design. Has written a master thesis in gamification.
	\item PhD candidate in computer science and health informatics.
	\item Master degree student in computer science.	
\end{itemize}

\begin{figure}[h!]
	\includegraphics[scale=0.2]{workshop220219}
		\caption {The participants on the workshop. The master student is absent as he is the photographer}
\end{figure}

The meeting started with the master student informing the status of the project by doing a cognitive walk-through and a demonstration of the application. 

The professor presented ideas for further development of the application. The important thing, was the concept of splitting up the questions in themes which relates to components in the treatment plan. The medical doctor helped identify these themes as assessment, diagnosis, management and treatment. Further we identified what type of questions we wanted to ask, and how they fits into different difficulty levels, based on the details of the questions. We ask factual questions for level 1. We use scenarios in level 2, where we apply facts and the detail level is categories. I.e. what class of medication should be administered to the patient. In level 3 we continue with scenarios, but here we ask for much more details, like the dosage of a medication or how often it should be administered. See figure \ref{table:QuestionsDetailLevels} to for an example of how the detail level gets higher as the student progress.

\begin{table}[h!]
	\label{table:QuestionsDetailLevels}
	\begin{tabular}{ | m{2em} | m{7em}| m{5.5em} | m{5em}| m{7.5em} |} 
		\hline
		Level & Question & Answer key & Distractions & Explanation \\
		\hline
		1 & Which of the following class of drugs is NOT indicated in the immediate management of severe asthma? & Inhaled corticosteroids & Short acting beta agonists, Oral corticosteroids, Anti-cholinergics & Inhaled corticosteroids are not indicated for acute management of asthma, they're indicated for long term control of symptoms \\
		\hline
		2 & Karen is found to be experiencing severe asthma and is given salbutamol. What other medications should she be given? & Prednisolone and oxygen & Ipratropium bromide, Formoterol & When  treating a patient with severe asthma, always start with oxygen, salbutamol and prednisolone. The evaluation of how the patient responds, will determine the next step of the treatment \\ 
		\hline
		3 & You diagnos Malin with severe asthma and initiate a treatment plan. What dose of salbutamol will you administer with the nebulizer? & 2.5mg & 3.5mg, 4.5mg, 5.5mg & In the scenario of severe asthma, nebulize 2.5mg of salbutamol \\
		\hline
	\end{tabular}
\caption{As the student progress to higher difficulty levels, the questions will ask for more details. Here we show the detail levels of Management}
\end{table}

When playing level 1 the student should get questions from all themes in level 1. When the student completes level 1, the student should no longer get questions from that theme. This is to avoid boring the user by repeating the questions the student already knows the answer of. He should only get questions from themes he struggles with. I.e. on the first run level 1, the user gets every question in assessment right, but have some mistakes with diagnosis and management. Then on the next run, he only gets questions from level 1 diagnosis and management. This continues until he has reached the passing condition on every theme in level 1. 

We further identified a dependency in the treatment plan. To be able to do a follow-up, the students first needs to know something about assessment, diagnosis and management of the patient. The follow-up is actually an evaluation of the treatment which have been given, based on the suggested diagnosis. The evaluation will tell how the patient responded to the treatment, and the student needs to take actions whether the patient responded or his condition became better or worse. When we have such a dependency, the student needs to complete assessment, diagnosis and management before follow-up gets unlocked. Since follow-up is only relevant in a situation where there has already been set a diagnosis and given a treatment, the follow-up is only part of level 2 and 3, where the questions are given as scenarios.

\begin{figure}[h!]
	\label{fig:FollowUpDependencies}
	\includegraphics[scale=0.6]{FollowUpDependencies}
		\caption {The student needs to acquire basic knowledge about assessment, diagnosis, management, before he can follow-up the treatment. This is a prerequisite for follow-up}
\end{figure}

To complete a level, all passing conditions at that level in each theme have to be met. When the student qualifies for a new level, he only gets questions from the level he plays. The same concept of only getting questions from the themes at that level you haven't met the passing condition, continues for level 2 and 3. 

We planned to have a visualization of the passing condition in the application. The passing condition will be shown in a chart in the summary section after each game. The passing condition will be marked as a line over every theme for the level the student plays. The students scores for each theme at that level be shown as bars. When a bar reaches the line, a passing condition is met.




\begin{figure}[h!]
	\includegraphics[scale=0.075]{workshop220219-2}
	\caption {Assessment and diagnosis are components in the treatment plan. In the learning map they are themes. Under each theme there are difficulty levels. Questions for each level are written on post-it notes}
\end{figure}

\begin{figure}[h!]
	\includegraphics[scale=0.075]{workshop220219-3}
	\caption {Management and follow-up are components in the treatment plan. In the learning map they are themes. Under each theme there are difficulty levels. Questions for each level are written on post-it notes}
\end{figure}

\begin{figure}[h!]
	\includegraphics[scale=0.075]{workshop220219-4}
	\caption {What type of questions the student will get at each level}
\end{figure}

The medical doctor continued the meeting by talking about the guidelines. The paediatric guideline of asthma is called "possible asthma". That is because in an emergency situation asthma is the most dangerous airway condition and can be lethal. If the patient shows signs of asthma, he will be treated for asthma to reduce risks of an unwanted scenario.

We identified users of the application: 
\begin{itemize}
	\item Formal training, where last year medical students are reading for their exams.
	\item Anyone can learn, so it can be used to inform and educate the public.
	\item In countries such as Kenya, where there are a large deficit in doctors and nurses, sometimes nurses has to work as doctors. Or community workers need to take the role as doctor or nurse. The application will help educating nurses and community workers for such scenarios.
\end{itemize}
 
 There was also talked about how the situation in medical training is for the student. When a patient comes to the emergency room with severe asthma, the medical doctors will have all their focus on that particular patient. The medical student will typically not take part in the assessment, diagnosis or the initial treatment of the patient. The medical student will typically only take part in the monitoring, evaluation and follow-up of the patient, when the situation is less critical. The application will give the medical student an alternative way to train in assessment, diagnosis and initial treatment of a made-up patient with severe asthma. 

The medical doctor continued the workshop by going through the Kenyan paediatric guideline of possible asthma \parencite{RepublicofKeny2016}. This is the guideline we will base our quiz on. The medical doctor answered questions from the group about details of the guideline. It is important to understand the general flow as well as the details to be able to make good questions for the quiz. The guidelines is poorly written in terms of wrong use of sentential operators. These mistakes needed to be clarified.

The rest of the workshop was for the participants brainstorming around the questions which will be used in the quiz. Each participant wrote questions on post-it notes, and placed them at a suiting level and theme on the blackboard. At the end, The medical doctor went through the questions and we had a small discussion around the suggested questions. We managed to produce question templates to be used in asthma quiz of the application.

\begin{figure}[h!]
	\includegraphics[scale=0.075]{workshop220219-5}
	\caption {Suggestions for questions were written on post-it notes and attached to a difficulty level under a theme on the black board}
\end{figure}




\chapter{Developing a learning tool for health workers}
\section{Extracting knowledge from the clinical  practice guidelines}
\section{Data models}
\subsection{DPF metamodeling}
\subsection{Entity model}
\subsection{Workflow model}
\subsection{Game model}
\section{Game engine}
As each question in a quiz are related to a certain component in the treatment plan or theme in the learning map, the student will be measured how well he performs on each of these themes. For the asthma guideline \parencite{RepublicofKeny2016}, we have identified four themes. Assessment where the student will be tested in the initial examination. Diagnosis, where the student will determine a diagnosis as well as the severity. Management, where the student will determine which actions should be done to treat and best give the best care to the patient. The last discipline is the follow-up, where the student will be tested in evaluating the treatment, give advise to and educate patient and caregivers, provide the right medication and regular follow-up.

By splitting up the score in themes, the student can easily see which areas he is strong and where he needs more training. 



We can also adapt the questions in each discipline to the student's level. If the student has proven to be very good in providing the right amount of medicine to asthma patient, we can provide more difficult questions to challenge the student some more. If he struggles at setting the right diagnose, we can provide more basic questions to strengthen the students basic knowledge. 

\textcolor{red}{The disciplines should be automatically picked from the entity (and worflow?) model.}

\textcolor{red}{The tree structure of discipline scores. Diagnosis have examination, investigation, setting the severity. Management have advises, medication, admit, surgery and so on.}




The student will also be provided with a total score, which will be the average score of each of the disciplines. The student can compare the total score of e.g. the asthma quiz and the jaundice quiz, and see which medical condition he needs to train more on.

\subsection{Multiple-try feedback}
% https://books.google.no/books?hl=en&lr=&id=GgCPAgAAQBAJ&oi=fnd&pg=PA125&dq=Interactive+with+multiple+tries+&ots=A9Z_BJS5t2&sig=RTv1FmOzU_qic9ADjDgKdJoHamU&redir_esc=y#v=onepage&q=Interactive%20with%20multiple%20tries&f=false
% https://onlinelibrary.wiley.com/doi/full/10.1348/000709905X39134
The quiz uses a concept which is called multiple-try feedback (MTC). That means for every question the student gets more than one attempt to get the answer right. A feedback will be given immediately after each answer is submitted. The feedback consists of a message which tells whether the answer is correct or wrong. If the answer is correct, the user will receive "correct" and an explanation of the answer. If the answer is wrong, there will be no hints or explanations than just "incorrect".

\begin{tabular}{ | m{10em} | m{6em}| m{6em} | m{6em} | m{5em} | } 
	\hline
	Concept & Abbreviation & Feedback after each question & Multiple attempts at each question & Hints on wrong answer \\ [0.5ex]
	\hline
No or delayed Feedback & NF or DF & No & No & No  \\
Knowledge of Correct Response & KCR  & Yes & No & No \\
Multiple-Try feedback with knowledge of Correct response  & MTC & Yes & Yes & No \\
Multiple-Try feedback with Hints & MTH & Yes & Yes & Yes \\
\hline
\end{tabular}

The point of doing MTC, is to make the student think over what was wrong with his first answer. Did the student misinterpreter the question? Was there a detail he missed? Does the student lack the knowledge or was he just sloppy in his first attempt?

\textcite{Clariana2006} did a study where they divided 82 students into five groups. DF-, KCR, MTC and two control groups. The first control group got a text and a question at the end. The second control group got a text, but there were no question given. After 5 days,post-test was held to see what the students had learned and remembered. The post-test questions were either identical to the questions in the learning material, transposed where the order of the stem of the question and the correct-response gets reversed, paraphrased where post-test questions had the identical content as the learning material, but the phrasing was different and used different words, and a combination of transpose and paraphrasing. The results showed that DF and KCR groups performed better on identical, transposed and paraphrased-transposed questions. MTC performed better on paraphrased questions. The conclusion was that DF and KCR was much better methods for remembering the learning material word for word, but MTC was better when you have to think and reason about what you have learned.

\textcite{Attali2015} further did a did a study on NF, KCR, MTC and MTH using open ended and multiple choice questions on mathematical problems. They showed that solving an open ended question rather than multiple choice was a more efficient way to learn. The learning outcome was the same for the students using NF and KCR. However the learning transfer was greater when using multiple-try (MTC), and even more so when getting a hint on incorrect answer (MTH). They explained the results effortfull and mindful problem solving. In a multiple-try feedback, the user will have to reflect on their errors, re-evaluate the problem and understand the initial error. An open ended question will also require more effort of the student, as they have to generate a an answer rather than selecting from alternatives. On the combination of multiple-try and multiple-choice, it was suggested that some users might be less likely to review their incorrect answer and mindlessly clicking on another alternative. 



According to \textcite{Morrison1995}, students which perform badly on answer until correct questions,  will often become frustrated, loose interest for reviewing the material and probably depress learning.

% From Attali2015, students might assosiate distractions with the scenario in multiple-choice, which is counterproductive when it comes to learning.


As thinking and reasoning about a diagnosis, treatment plan, evaluation and follow-up of a treatment is part of a medical procedure, we believe that multiple-try feedback is the right approach. Because of the nature of a mobile app, where gestures are more convenient than typing sentences, multiple-choice seems to be the right choice even, though open ended questions has proven better results in. There's also a technical problem with evaluating free typed sentences.

Some of the questions in the app are too simple for a hint to be meaningful.Example: "the symptoms for asthma is" and the answer can be "cough and wheeze". Where hinting "cough", would be giving away the answer, especially in a multiple-choice format. However, the data model supports hints as links to external learning material. E.g. the student could look for the answer in the guideline itself.

We solved the "answer until correct"-problem described by \textcite{Morrison1995}, by having a "read more" button displayed upon incorrect answer. The "read more"-button will display the correct answer, an explanation and continue to the next question. Avoiding the user becoming frustrated and discouraged by having to brute-force the answer keys to progress.




\subsection{Reward system}
% SHOW ANSWER
By having multiple-try feedback, another problem rises, and that is the reward system. If there is no penalty for incorrect answers, a student which needs ten attempts per questions, will get the same score as a student which answers all the questions correctly on the first attempt.

\textcite{Attali2015} solved the problem by giving 1 point for answering correctly on the first attempt. 2/3 points for the second attempt, 1/3 for the third and 0 points if the third attempt was incorrect. A limitation with this method is that it makes no sense for the student to make more than three attempts. \textcite{Morrison1995} had another strategy where they adjust the scores by dividing the total score by the total number of attempts during the quiz. A consequence is that attempt number two will have a huge penalty which is halving the students total score. While attempt number twenty will give a very small penalty from attempt nineteen.

The solution we used was having a fixed value for every answer alternative. The quiz author chooses the penalty for each distraction and reward for each answer key. The idea is that the distractions can have some sort of degree of wrong or right, and this can be reflected in the scoring. On the question "what are the symptoms of asthma?", "difficulty breathing" is a more correct answer than "fever", as "difficulty breathing" is a symptom of asthma in combination with wheeze. Fever is not an asthma symptom at all. In future work, the penalties can be automated as you can see from the entity model whether the symptom belongs to the asthma guideline or not. A distraction from respiratory disorders may give a larger penalty than a distraction from the asthma guideline, but smaller penalty than symptoms not belonging to respiratory diseases.

Both \textcite{Attali2015} and \textcite{Morrison1995} avoids the scenario where the user gets a total minus score. This may be a strength of these methods, as a negative total score seems like a very harsh feedback and might demotivate the student. In our solution we use negative numbers as penalties on distractions, such that a negative total score may happen. We try to limit the likelihood of a negative score by providing a very reward for a correct answer and a very small penalty for a distraction. Typically the reward is 10 points and the penalty -1 og -2 points. The intention is to encourage the student to review the incorrect answer and try again. As the format is multiple-choice and the penalty-reward ratio, there is a little risk involved trying multiple times. But giving up by clicking "learn more", the student will not get an additional penalty, but will miss out on the reward. By clicking answer alternatives mindlessly and consequently clicking "learn more" will probably not end up in a negative score, but is more likely to end up in a negative score than mindlessly click answer alternatives until correct.

%----------------------------------------------------------------------------------


%Each question will have several answer alternatives the student can choose from. Each answer alternative will have a reward or penalty related to them. The correct answer will have a great reward, while wrong answers will have a small penalty. The quiz author will have the opportunity to specify the rewards, such that he can give even smaller penalties for partly correct answers. The idea of the reward- penalty system is to increase learning. if the student answers wrong the first time, he will be given the possibility to reflect over the question once more or perhaps read the guideline to learn before he commits his second attempt. We are aware that providing a minus score for making an attempt can be very demotivating, but it is to avoid the situation where a student gets the same (or better) score for making ten attempts than only needing one attempt. A small penalty will have a very small impact when the reward per question is high, but in situations where the student performs very poorly and ends with a negative total score, it is possible to adjust this to a small positive score on presentation for the student. Not giving a too harsh feedback for trying to learn.



\textcolor{red}{A solution to having a not very strict game, encouraging to playing and learning, one can also have a very strict examination version. The idea is that after examination, the results will be sent to the lecturer (or a governing body of some kind) to evaluate what the overall knowledge of the students, as well as details of what the students are really good in and where do they struggle. The lecturer can then target the weak of points of the students in one of the next lectures. }


\subsection{Unlocking harder levels at a certain category}
\textcolor{red}{(somewhere in the paper I need to refer to Eides, Kristensens and Lamos paper, and discuss the knowledge, learning and student maps and that they need to prove basic knowledge in some disciplines before they can unlock content in other disciplines.)}
\subsection{Visualization of game statistics}
\subsection{Automatically generating new questions}
\section{The mobile application}
\begin{itemize}
	\item React
	\item React-Native
	\item React-Native-Navigation (Wix)
	\item Redux
	\item React-Redux
	\item Redux-Thunk
	\item Highcharts
	\item Jest
\end{itemize}
\subsection{React-Native and Redux}
\subsection{User interface and flow of the user interaction}
\section{Architecture of the whole system}
\subsection{Visualization}
\section{Evaluation}



\chapter{Discussion}
\section{Research questions}
\section{Limitations of the model}
\begin{itemize}
	\item Can't ask questions like "what are the symptoms for severe asthma?"
	\item Difficult to ask what NOT to do. If the vertex doesn't exist, only an empty string gets returned. Can only be used were we actually have written "don't admit to the hospital" as an example with hospitalization.
	\item The inheritance makes it difficult to generalize some questions. We can't make a template which asks about the Rate a medicine should be taken with. We need to specifically ask for that medicine. To be able to ask for a general medicine, one solution can be to introduce a new tag which compares the substring of the type of the vertex. Another solution is to use the meta model and not the instance model. We don't use inheritance on diagnosis because of this.
	\item To avoid the problem described in the previous point, we don't use inheritance on Diagnosis. A limitation here is that  
	a patient can only have one diagnosis.
\end{itemize}
\section{Observations}
\section{Challenges}
\section{Reflection}



\chapter{Conclusions}
\section{Further research and development}


\backmatter
% bibliography, glossary and index would go here.
\end{document}