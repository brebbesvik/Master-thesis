
\section{Clinical Practice Guidelines}
\textcite{Fervers2010} claims that for clinicians, increased medical knowledge is associated with an exponential growth of scientific data and published material. It is impossible to keep up, as well as integrating all the new information into daily practice to give patients the best possible care.  \textcite{Masic2008} gives an example where a general practitioner should read 19 articles per day to keep up with the new medical information, while only having time for reading one hour per week. Reading 19 articles per day, would acquire more time than the clinician has available for treating patients. This problem is known as academic isolation \parencite{Masic2008}.

Evidence Based Medicine (EBM) suggests that instead of routinely reading dozens of articles, the clinicians should target their reading to specific patient problems. Developing clinical questions and then searching for the answer (problem based approach) may be a more productive way to keep up with the new medical knowledge \parencite{Masic2008}. The EBM definition further puts an emphasize on integrating the best evidence in decision making with the clinicians expertise and the patients values and expectations \parencite{Masic2008}. 

The concept of EBM is about transferring knowledge from clinical research into clinical practice, and Clinical Practice Guidelines (CPG) can play an instrumental role in this process \parencite{Fervers2010}.

The Institute of Medicine (IOM) has given the following definition of clinical practice guidelines: "CPGs are statements that include recommendations intended to optimize patient care. These statements are informed by a systematic review of evidence and an assessment of the benefits and costs of alternative care options" \parencite{Guidelines2011}.

The definition given by IOM covers the goals in EBM, and also takes the cost into account. In fact, \textcite{Clayton1995} have shown that in some situations good use of appropriate guidelines and protocols can reduce as much as 25\% of the cost of healthcare.

% I can use Woolfs article and write even more about benefits. Should probably do that.

Even though the CPGs have proven to improve the quality of health care while reducing practice variability and the cost of patient care \parencite{DeClercq2008}, it is well recognized that CPGs have had a limited effect on changing the clinicians practice methods. \textcite{Cabana1999} lists the following reasons:
\begin{itemize}
	\item \textbf{Lack of awareness:} the clinician is not aware of the guideline's existence.
	\item \textbf{Lack of familiarity:} the clinician is not familiar with the content of the guideline.
	\item \textbf{Lack of agreement:} the clinician had various reasons to disagree with the guideline, such as they are oversimplified, disagree with the evidence or not worth the patient risk, discomfort or cost.
	\item \textbf{Lack of self-efficacy:} is the lack of self-confidence in that the clinician can execute the recommendations of the guideline correctly.
	\item \textbf{Lack of outcome expectancy:} the clinician doesn't believe the outcome of the recommended treatment will meet the outcome expectancy.
	\item \textbf{Inertia of previous practice:} the custom, habit or previous training can hinder the adaptation of clinical practice.
	\item \textbf{External Barriers:} the guidelines are not easy to use, not convenient, cumbersome and confusing.
\end{itemize}One example of an external barrier is the Guidelines for the Diagnosis and Management of Asthma \parencite{NationalHeartLungandBloodInstitute2007}, which consists of 440 pages. Such a large document is not convenient to use at the point of care. According to \textcite{Shortliffe1998}, CPGs in monographs and journal articles tend to sit on book shelves at the time their knowledge could prove the most valuable to the clinicians. Lack of awareness in other words.

\subsection{Discussion}
According to \textcite{Woolf1999}, clinicians sometimes have good reasons to disagree with some of the content of a guideline. \textcite{Woolf1999} points out three reasons:
\begin{enumerate}
	\item The scientific evidence of the recommendation can be lacking, misleading or misinterpreted.
	\item The recommendations may be influenced by the authors. What the authors believe, may be inferior to other options, ineffective or harmful.
	\item As the guideline may be written to control cost, serve societal needs or protect special interest, the recommendations may be suboptimal for the patient.
\end{enumerate}
There exists grading systems which grade the quality of evidence and strength of recommendations. GRADE is such a grading system \parencite{Guyatt2008}. When displaying guidelines to clinicians, it is a strong point to display the grade of evidence, as the clinician has to choose between several treatment options.


%\textcolor{red}{\begin{itemize}
%	\item Medical knowledge increases. Hard to keep track
%	\item Guidelines is a summary of the available evidence of the medical conditions and provide management and recommendations
%	\item A well-developed guideline reduces
%	variations in care, improves diagnostic accuracy,
%	promotes effective therapy and discourages ineffective
%	therapies all which contribute to improved
%	quality of care (citation)
%	\item The CPGs are not used enough
%	\item Dissemination and implementation
%	\item Large volume of excisting guidelines. Difficult to use at the point of care
%	\item Dissemination
%	\item Different practice even in the same country
%\end{itemize}}


\section{Serious games}
When searching the literature for the definition of serious games, there seem to be many different understandings of what serious games really is. However, these definitions seem to have the common understanding that serious games are games which are used for other purposes than just pure entertainment \parencite{Susi2015}. This is actually a very broad category, where we can find games which are used to test job applicants or to improve our health by encouraging us stay more active.

\textcite{Michael2006} defines serious games as "a serious game in which education (in its various forms) is the primary goal, rather than entertainment". \textcite{Michael2006} emphasizes that education and entertainment should not be in conflict, but that they can overlap. The feeling of learning something new or getting better at something, can be quite satisfying and can serve as a motivating factor.

Serious games also have the advantage over educational books and movies that the student can demonstrate and apply what he has learnt, through tasks in the game \parencite{Michael2006}.    Serious games seem more effective than training with conventional instruction methods, as the knowledge gains persists in the long term memory, and the learner can build on this well-structured prior knowledge through his learning career \parencite{Wouters2013}. Serious games, however, seems to be most effective when they are supplemented with instructional learning methods. Not only do the student get to learn by doing, but he also gets the opportunity to reflect over what he has learnt and to verbalize the new knowledge, making it easier to integrate it into his knowledge base \parencite{Wouters2013}. 





\section{Motivation}
By making a serious game for clinical practice guideline training, we can address some of the reasons why the CPGs haven't had a greater impact on clinicians practice methods \parencite{Cabana1999}:
\begin{itemize}
	\item \textbf{Lack of awareness:} The more projects there are around CPGs, the more focus will they get and more people may get aware of their existence. By making a serious game, we may be able to target some user groups which where hard to reach in traditional ways. 
	\item \textbf{Lack of familiarity:} By playing or working with the game, the student might learn more about the content and will become familiar with the CPGs. The student may also be encouraged to study the CPGs in the traditional ways after having played the game.  
	\item \textbf{Lack of self-efficacy:} By repeatedly solving practical tasks in the game, the student may become confident in that they are capable of executing the treatment recommended by a CPG.
	\item \textbf{External barriers:} convenient, cumbersome and confusing CPGs will by approach be converted to a game format. Even though a game isn't a good encyclopaedia at the point of care, for some user groups a game might be a better format for studying. Such a combination of instructional learning methods and serious games have shown positive learning results \parencite{Wouters2013}. Having built well-structured prior knowledge through the use of serious CPG games may also help at the point of care.
\end{itemize}

Another motivational reason for making a serious game is the scalability. How can we best train 10, 100 or 1000 clinicians in the best practices of medical guidelines? There are logistics problems with instructional courses and training sessions, such as cost of money and time, and there's a practical limit for how many attendees can attend a course at the same time. A mobile game scales much better as downloading an mobile application is much cheaper, and it can be played almost anywhere at any time. There's no limitation on how many playing participants.

\subsection{Asthma}
Asthma is a repository disease, which in the recent years have had an almost exponential growth rate among children in Oslo. From 0.4\% in the first Norwegian report in 1948, to 8\% in 1993 and 20.2\% in 2006. Similar results were found in the rest of Norway in the early 90s \parencite{Carlsen2006}. 

Asthma growth amongst children is not only an issue in Norway. According to \textcite{Odhiambo1998}, 3\% of children in rural areas in Kenya had asthma in 1998 and 9.5\% of the children in urban areas. Before this study it was a claim that asthma among African children was rare, which is no longer true \parencite{Odhiambo1998}.

As we can see, there is an exponential growth in the numbers of children being diagnosed with asthma both in the western world African countries. Consequently it is urgent to find answers to prevent further increase of asthma amongst children in the years to come \parencite{Carlsen2006}.

This thesis aims at contributing in addressing those global problems through exploring the potential of developing and using a serious CPG game to empower health care workers in diagnosing and treating children possibly suffering from asthma.


\section{Summary}
Clinical practice guidelines are evidence based statements, which includes recommendations to optimize patient care. Well defined guidelines improve the quality of health care at a lower cost, as well as reduce practice variability.

However, clinical practice guidelines have had an limited effect in changing clinician practice methods. In this thesis we will develop a serious game to address some of the reasons why the clinical practice guidelines haven't been put into more use.

Our work will be based around paediatric asthma as a contribution to put a focus on the the dramatically increase of asthma in children.