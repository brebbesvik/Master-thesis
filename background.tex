\chapter{Background}
\section{Clinical Practice Guidelines}
For a clinician, \textcite{Fervers2010} claims that increased medical knowledge is associated with an exponential growth of scientific data and published material. It is impossible to keep up, as well as integrating all the new information into daily practice to give patients the best possible care.  \textcite{Masic2008} gives an example where a general practitioner should read 19 articles per day to keep up with the new medical information, while only having time to read one hour per week. The clinician will spend more time reading articles than treating patients. This problem is known as academic isolation.

Evidence Based Medicine (EBM) suggests that instead of routinely reading dozens of articles, the clinicians should target their reading to specific patient problems. Developing clinical questions and then searching for the answer (problem based approach), may be a more productive way to keep up with the new medical knowledge \parencite{Masic2008}. The EBM definition further puts an emphasize on integrating the best evidence in decision making with the clinicians expertise and the patients values and expectations \parencite{Masic2008}. 

The concept of EBM is about transferring knowledge from clinical research into clinical practice, and Clinical Practice Guidelines (CPG) can play an instrumental role in this process \parencite{Fervers2010}.

The Institute of Medicine (IOM) has given the following definition of clinical practice guidelines: "CPGs are statements that include recommendations intended to optimize patient care. These statements are informed by a systematic review of evidence and an assessment of the benefits and costs of alternative care options" \parencite{Guidelines2011}

The definition given by IOM covers the goals in EBM, and also takes the cost into account. In fact, \textcite{Clayton1995} have shown that in some situations good use of appropriate guidelines and protocols can reduce as much as 25\% of the cost of healthcare.

% I can use Woolfs article and write even more about benefits. Should probably do that.

Even though the CPGs have proven to improve the quality of health care while reducing practice variability and the cost of patient care \parencite{DeClercq2008}, it is well recognized that CPGs have had a limited effect on changing the clinicians practice methods. \textcite{Cabana1999} lists the following reasons:
\begin{itemize}
	\item Lack of awareness.
	\item Lack of familiarity.
	\item Lack of agreement with the content.
	\item Lack of self-efficacy.
	\item Lack of outcome expectancy.
	\item Inertia of previous practice. 
	\item External Barriers; the guidelines are not easy to use, not convenient, cumbersome and confusing.
\end{itemize}One example of external barrier is the Guidelines for the Diagnosis and Management of Asthma \parencite{NationalHeartLungandBloodInstitute2007}, which consists of 440 pages. Such a large journal is not convenient to use at the point of care. According to \textcite{Shortliffe1998}, CPGs in monographs and journal articles tend to sit on book shelves at the time their knowledge could prove the most valuable to the clinicians. 

According to \textcite{Woolf1999}, clinicians sometimes have good reasons to disagree with some of the content of a guideline. \textcite{Woolf1999} points out three reasons:
\begin{enumerate}
	\item The scientific evidence of the recommendation can be lacking, misleading or misinterpreted.
	\item The recommendations may be influenced by the authors. What the authors believe, may be inferior to other options, ineffective or harmful.
	\item As the guideline may be written to control cost, serve societal needs or protect special interest, the recommendations may be suboptimal for the patient.
\end{enumerate}
There exists grading systems which grade the quality of evidence and strength of recommendations. GRADE is such a grading system \parencite{Guyatt2008}.


\textcolor{red}{\begin{itemize}
	\item Medical knowledge increases. Hard to keep track
	\item Guidelines is a summary of the available evidence of the medical conditions and provide management and recommendations
	\item A well-developed guideline reduces
	variations in care, improves diagnostic accuracy,
	promotes effective therapy and discourages ineffective
	therapies all which contribute to improved
	quality of care (citation)
	\item The CPGs are not used enough
	\item Dissemination and implementation
	\item Large volume of excisting guidelines. Difficult to use at the point of care
	\item Dissemination
	\item Different practice even in the same country
\end{itemize}}
\section{Serious games}
\section{Asthma}
\subsection{Challenges}
\section{Related work}
\section{Summary}